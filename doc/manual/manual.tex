\RequirePackage[l2tabu, orthodox]{nag}		% detailed warnings and complaints
\documentclass[	11pt,
				a4paper,
				abstract=true,
				twoside=true,
				bibliography=totoc, 
				headinclude=true,
				footinclude=false]{scrartcl}

%\newif\iftwoSide\twoSidetrue
\newif\iftwoSide\twoSidefalse

\usepackage[utf8]{inputenc}
\usepackage[T1]{fontenc}
\usepackage{scrhack}						% KOMA script fixes
\usepackage{amsmath}						% essential AMS packages
\usepackage{amsthm}							% essential AMS packages
\usepackage{amssymb}						% essential AMS packages
\usepackage{mathtools}						% extensions for amsmath
\usepackage{xcolor}							% easy color specification
\usepackage{graphicx}						% handling of includegraphics
\usepackage{longtable}						% tables with pagebreaks
\usepackage{booktabs}						% lines in tables
\usepackage[ngerman,english]{babel}			% letztgenannte Sprache ist Default
\usepackage{csquotes}						% ensure that quotation marks are set correct according to language
\usepackage{ellipsis}						% corrects whitespace around \dots{}
\usepackage{fixltx2e}						% patches for Latex
\usepackage{geometry}						% page layout
%FONT
\usepackage{mathptmx}							% Times (serif default)
\usepackage[scaled=.90]{helvet}					% Helvetica (sans serif default)
\DeclareMathAlphabet{\mathcal}{OMS}{cmsy}{m}{n}	% use standard mathcal font
\renewcommand*\ttdefault{txtt}					% TXTT monospace
\usepackage{microtype}							% better spacing (might be font-dependent!)

%%%%%%%%%%%%%%%%%%%%%%%%%%%%%%%%%%%%%%%%%%%%%%%%%%%%%%%%%%%%%%%%%%%%%%72
%	LISTINGS PACKAGE
%%%%%%%%%%%%%%%%%%%%%%%%%%%%%%%%%%%%%%%%%%%%%%%%%%%%%%%%%%%%%%%%%%%%%%72
\usepackage{listings}						% for displaying source code
\lstdefinestyle{myC}{
   belowcaptionskip=1\baselineskip,
   breaklines=true,
   frame=L,
   xleftmargin=\parindent,
   language=C,
   showstringspaces=false,
   basicstyle=\footnotesize\ttfamily\upshape,
   identifierstyle=\ttfamily\upshape,
   commentstyle=\rmfamily\itshape,
   stringstyle=\ttfamily\itshape,
   keywordstyle=\ttfamily\upshape\bfseries,
%   identifierstyle=\color{blue},
%   commentstyle=\itshape\color{purple!40!black},
%   stringstyle=\color{orange},
%   keywordstyle=\bfseries\color{green!40!black},
}
\lstdefinestyle{myFort}{
   belowcaptionskip=1\baselineskip,
   breaklines=true,
   frame=L,
   xleftmargin=\parindent,
   language=[90]Fortran,
   showstringspaces=false,
   basicstyle=\footnotesize\ttfamily\upshape,
   identifierstyle=\ttfamily\upshape,
   commentstyle=\rmfamily\itshape,
   stringstyle=\ttfamily\itshape,
   keywordstyle=\ttfamily\upshape\bfseries,
}

%%%%%%%%%%%%%%%%%%%%%%%%%%%%%%%%%%%%%%%%%%%%%%%%%%%%%%%%%%%%%%%%%%%%%%72
%	ALGORITHMS PACKAGE
%%%%%%%%%%%%%%%%%%%%%%%%%%%%%%%%%%%%%%%%%%%%%%%%%%%%%%%%%%%%%%%%%%%%%%72
\usepackage[boxruled]{algorithm2e}			% algorithm environment
\SetAlgoLined								% algorithm layout with vertical lines
\DontPrintSemicolon							% don't print semicolon at line end
\makeatletter								% badboxes for boxruled algorithms
\renewcommand{\algocf@caption@boxruled}{%
  \hrule
  \hbox to \hsize{%
    \vrule\hskip-0.4pt
    \vbox{   
       \vskip\interspacetitleboxruled%
       \unhbox\algocf@capbox\hfill
       \vskip\interspacetitleboxruled
       }%
     \hskip-0.4pt\vrule%
   }\nointerlineskip%
}%
\makeatother

%%%%%%%%%%%%%%%%%%%%%%%%%%%%%%%%%%%%%%%%%%%%%%%%%%%%%%%%%%%%%%%%%%%%%%72
%	SETTINGS AND LAYOUT OPTIONS
%%%%%%%%%%%%%%%%%%%%%%%%%%%%%%%%%%%%%%%%%%%%%%%%%%%%%%%%%%%%%%%%%%%%%%72

%	KOMA layout options
\setkomafont{disposition}{\bfseries}	% headlines bold with serifs
\raggedbottom					% allow different page heights (fixed distance between paragraphs)
\setcounter{tocdepth}{1}
%\pagestyle{headings}

\iftwoSide % equal to DIV 10 with smaller bottom margin
\KOMAoptions{twoside=true}
\geometry{a4paper,twoside,left=20mm,top=30mm,right=40mm,bottom=50mm,bindingoffset=10mm,includehead}
\else % for one-sided version, same margin width left and right
\KOMAoptions{twoside=false}
\geometry{a4paper,left=30mm,top=30mm,right=30mm,bottom=50mm,includehead}
\fi


% always include hyperref package as the last package because it conflicts with other packages!
\usepackage[bookmarks]{hyperref}

%%%%%%%%%%%%%%%%%%%%%%%%%%%%%%%%%%%%%%%%%%%%%%%%%%%%%%%%%%%%%%%%%%%%%%72
% MY MACROS
%%%%%%%%%%%%%%%%%%%%%%%%%%%%%%%%%%%%%%%%%%%%%%%%%%%%%%%%%%%%%%%%%%%%%%72

% Sets
\newcommand{\R}{\mathbb{R}}				% real numbers
\newcommand{\N}{\mathbb{N}}				% natural numbers
\newcommand{\Id}{\mathbb{I}}			% identity
\newcommand{\A}{\mathcal{A}}			% active set
\newcommand{\F}{\mathcal{F}}			% filter
\renewcommand{\S}{\mathcal{S}}			% stable active set (``set S'')
\newcommand{\Ws}{\mathcal{W}}			% QP working set

% Optimization
\renewcommand{\L}{\mathcal{L}}
\newcommand{\asit}{\nu}
%\newcommand{\asit}{\iota}
\newcommand{\st}{\textup{s.t.}}

% Command
\newcommand{\norm}[1]{\left\lVert#1\right\rVert}	% w scaled lines
\newcommand{\normn}[1]{\lVert#1\rVert}				% w/o scaled lines
\newcommand{\abs}[1]{\left\lvert#1\right\rvert}		% w scaled lines
\newcommand{\absn}[1]{\lvert#1\rvert}				% w/o scaled lines
\DeclareMathOperator{\diag}{diag}

% Software
\newcommand{\qpOASES}{\texttt{qpOASES}}
\newcommand{\blockSQP}{\texttt{blockSQP}}
\newcommand{\method}{\texttt{SQPmethod}}
\newcommand{\options}{\texttt{SQPoptions}}
\newcommand{\stats}{\texttt{SQPstats}}
\newcommand{\problem}{\texttt{ProblemSpec}}
\newcommand{\init}{\texttt{init}}
\newcommand{\evaluate}{\texttt{evaluate}}
\newcommand{\reduce}{\texttt{reduceConstrVio}}
\newcommand{\heu}{r}

%%%%%%%%%%%%%%%%%%%%%%%%%%%%%%%%%%%%%%%%%%%%%%%%%%%%%%%%%%%%%%%%%%%%%%72
% END HEADER
%%%%%%%%%%%%%%%%%%%%%%%%%%%%%%%%%%%%%%%%%%%%%%%%%%%%%%%%%%%%%%%%%%%%%%72


\title{\blockSQP\ user's manual}
\author{Dennis Janka}

\begin{document}
\maketitle
\tableofcontents
\clearpage
%%%%%%%%%%%%%%%%%%%%%%%%%%%%%%%%%%%%%%%%%%%%%%%%%%%%%%%%%%%%%%%%%%%%%%72
\section{Introduction}
%%%%%%%%%%%%%%%%%%%%%%%%%%%%%%%%%%%%%%%%%%%%%%%%%%%%%%%%%%%%%%%%%%%%%%72
\blockSQP\ is a sequential quadratic programming method for finding local solutions
of nonlinear, nonconvex optimization problems. It is particularly suited for
---but not limited to---problems whose Hessian matrix has block-diagonal
structure such as problems arising from direct multiple shooting
parameterizations of optimal control or optimum experimental design problems.

\blockSQP\ has been developed around the quadratic programming solver
\qpOASES~\cite{Ferreau2013} to solve the quadratic subproblems. Gradients of the objective
and the constraint functions must be supplied by the user in sparse or dense format. 
Second derivatives are approximated by a combination of SR1 and BFGS updates. 
Global convergence is promoted by the filter line search of Waechter and Biegler~\cite{Waechter2005b,Waechter2005}
that can also handle indefinite Hessian approximations.

The method is described in detail in \cite[Chapters 6--8]{Janka2015}. These chapters are largely self-contained. The notation used throughout this manual is the same as in~\cite{Janka2015}. A publication~\cite{Janka2015b} is currently under review.

%%%%%%%%%%%%%%%%%%%%%%%%%%%%%%%%%%%%%%%%%%%%%%%%%%%%%%%%%%%%%%%%%%%%%%72
\section{Installation}
%%%%%%%%%%%%%%%%%%%%%%%%%%%%%%%%%%%%%%%%%%%%%%%%%%%%%%%%%%%%%%%%%%%%%%72
The following steps

\begin{enumerate}
\item Download and install qpOASES from \url{https://projects.coin-or.org/qpOASES}.

	It is recommended to use at least release 3.2.0. Alternatively, check out revision 155 from the \qpOASES\ subversion repository that is located at \url{https://projects.coin-or.org/svn/qpOASES/trunk/}. For best performance it is strongly recommended to install the sparse solver \texttt{MA57} from HSL as described in the \qpOASES\ manual, Sec. 2.2.
\item In the \blockSQP\ main directory, open \texttt{makefile} and set \texttt{QPOASESDIR} to the correct location of the \qpOASES\ installation.
\item Compile \blockSQP\ by calling \texttt{make}. This should produce a shared library \texttt{libblockSQP.so} in  \texttt{lib/}, as well as executable example problems in the \texttt{examples/} folder.
\end{enumerate}


%%%%%%%%%%%%%%%%%%%%%%%%%%%%%%%%%%%%%%%%%%%%%%%%%%%%%%%%%%%%%%%%%%%%%%72
\section{Setting up a problem}
%%%%%%%%%%%%%%%%%%%%%%%%%%%%%%%%%%%%%%%%%%%%%%%%%%%%%%%%%%%%%%%%%%%%%%72
A nonlinear programming problem (NLP) of the form
\begin{subequations}\label{eq:nlp}
\begin{align}
\min_{x\in\R^{n}}\ &\varphi(x) \\
\st\ & b_{\ell} \leq \begin{bmatrix} x\\c(x)\end{bmatrix} \leq b_{u}
\end{align}
\end{subequations}
is characterized by the following information that must be provided by the user:
\begin{itemize}
\item The number of variables, $n$,
\item the number of constraints, $m$,
\item the objective function, $\varphi:\R^{n}\longrightarrow\R$,
\item the constraint function, $c:\R^{n}\longrightarrow\R^{m}$,
\item and lower and upper bounds for the variables and constraints, $b_{\ell}$ and $b_{u}$.
\end{itemize}
In addition, \blockSQP\ requires the evaluation of the
\begin{itemize}
\item objective gradient, $\nabla \varphi(x)\in\R^{n}$, and the
\item constraint Jacobian, $\nabla c(x)\in\R^{m\times n}$.
\end{itemize}
Optionally, the following can be provided for optimal performance of \blockSQP:
\begin{itemize}
\item In the case of a block-diagonal Hessian, a partition of the variable vector $x$ corresponding to the diagonal blocks,
\item a heuristic function $r$ to compute a point $x$ where a reduced infeasibility can be expected, $\heu:\R^{n}\longrightarrow\R^{n}$.
\end{itemize}

\blockSQP\ is written in C++ and uses an object-oriented programming paradigm. The method itself is implemented in a class \method. Furthermore, \blockSQP\ provides a basic class \problem\ that is used to specify an NLP of the form~\eqref{eq:nlp}. To solve an NLP, first an instance of \problem\ must be passed to an instance of \method. Then, \method's appropriate methods must be called to start the computation.

In the following, we first describe the \problem\ class and how to implement the mathematical entities mentioned above. 
Afterwards we describe the necessary methods of the \method\ class that must be called from an appropriate driver routine. Some examples where NLPs are specified using the \problem\ class and then passed to \blockSQP\ via a simple C++ driver routine can be found in the \texttt{examples/} subdirectory.


%---------------------------------------------------------------------72
\subsection{Class \problem}
%---------------------------------------------------------------------72
Dense and sparse problems

Implement constructor

Variables must be set
%---------------------------------------------------------------------72
\subsubsection{Function \init}
%---------------------------------------------------------------------72


%---------------------------------------------------------------------72
\subsubsection{Function \evaluate}
%---------------------------------------------------------------------72


%---------------------------------------------------------------------72
\subsubsection{Function \reduce}
%---------------------------------------------------------------------72

%---------------------------------------------------------------------72
\subsection{Class \method}
%---------------------------------------------------------------------72

%%%%%%%%%%%%%%%%%%%%%%%%%%%%%%%%%%%%%%%%%%%%%%%%%%%%%%%%%%%%%%%%%%%%%%72
\section{Options and parameters}
%%%%%%%%%%%%%%%%%%%%%%%%%%%%%%%%%%%%%%%%%%%%%%%%%%%%%%%%%%%%%%%%%%%%%%72
In this section we describe all options that are passed to \blockSQP\ through the \texttt{SQPoptions} class.
We distinguish between algorithmic options and algorithmic parameters. The former are used to choose between different algorithmic alternatives, e.g., different Hessian approximations, while the latter define internal algorithmic constants. As a rule of thumb, whenever you are experiencing convergence problems with \blockSQP, you should try different algorithmic options first before changing algorithmic parameters.

Additionally, the output can be controlled with the following options:
\begin{longtable}[c]{lll}
Name 							& Description/possible values									& Default	\\\hline\hline
%
\texttt{printLevel}				& Amount of onscreen output per iteration						& 1			\\
								& 0: no output													&			\\
								& 1: normal output												&			\\
								& 2: verbose output												&			\\\hline
%								
\texttt{printColor}				& Enable/disable colored terminal output						& 1			\\
								& 0: no color													&			\\
								& 1: colored output in terminal									&			\\\hline
%
\texttt{debugLevel}				& Amount of file output per iteration							& 0			\\
								& 0: no debug output											& 			\\
								& 1: print one line per iteration to file 						& 			\\
								& 2: extensive debug output to files (impairs performance)		& 			\\\hline
\end{longtable}

%---------------------------------------------------------------------72
\subsection{List of algorithmic options}
%---------------------------------------------------------------------72
\begin{longtable}[c]{lll}
Name 							& Description/possible values									& Default	\\\hline\hline
%
\texttt{sparseQP}				& \qpOASES\ flavor												& 2			\\
								& 0: dense matrices, dense factorization of red. Hessian		&			\\
								& 1: sparse matrices, dense factorization of red. Hessian		&			\\
								& 2: sparse matrices, Schur complement approach					&			\\\hline
%
\texttt{globalization}			& Globalization strategy										& 1			\\
								& 0: full step 													&			\\
								& 1: filter line search globalization							&			\\\hline
%
\texttt{skipFirstGlobalization}	& 0: deactivate globalization for the first iteration			& 1			\\
								& 1: normal globalization strategy in the first iteration		&			\\\hline
%
\texttt{restoreFeas}			& Feasibility restoration phase									& 1			\\
								& 0: no feasibility restoration phase 							&			\\
								& 1: minimum norm feasibility restoration phase					&			\\\hline
%
\texttt{hessUpdate}				& Choice of first Hessian approximation							& 1			\\
								& 0: constant, scaled diagonal matrix							&			\\
								& 1: SR1														&			\\
								& 2: BFGS														&			\\
								& 3: [not used]													&			\\
								& 4: finite difference approximation							&			\\\hline
%
\texttt{hessScaling}			& Choice of scaling/sizing strategy for first Hessian			& 2			\\
								& 0: no scaling													&			\\
								& 1: scale initial diagonal Hessian with $\sigma_{\textup{SP}}$	&			\\
								& 2: scale initial diagonal Hessian with $\sigma_{\textup{OL}}$	&			\\
								& 3: scale initial diagonal Hessian with $\sigma_{\textup{Mean}}$	&		\\
								& 4: scale Hessian in every iteration with $\sigma_{\textup{COL}}$	&		\\\hline
%
\texttt{fallbackUpdate}			& Choice of fallback Hessian approximation						& 2			\\
								& (see \texttt{hessUpdate})										&			\\\hline
%
\texttt{fallbackScaling}		& Choice of scaling/sizing strategy for fallback Hessian		& 4			\\
								& (see \texttt{hessScaling})									&			\\\hline
%
\texttt{hessLimMem}				& 0: full-memory approximation									& 1			\\
								& 1: limited-memory approximation								&			\\\hline
%
\texttt{blockHess}				& Enable/disable blockwise Hessian approximation				& 1			\\
								& 0: full Hessian approximation									&			\\
								& 1: blockwise Hessian approximation							&			\\\hline

%
\texttt{hessDamp}				& 0: enable BFGS damping										& 1			\\
								& 1: disable BFGS damping										&			\\\hline
%
\texttt{whichSecondDerv}		& User-provided second derivatives								& 0			\\
								& 0: none														&			\\
								& 1: for the last block											&			\\
								& 2: for all blocks (same as \texttt{hessUpdate=4})				&			\\\hline
\texttt{maxConvQP}				& Maximum number of convexified QPs (\texttt{int}>0)			& 1			\\\hline
%
\texttt{convStrategy}			& Choice of convexification strategy							& 0			\\
								& 0: Convex combination between									&			 \\
								& \phantom{0: }\texttt{hessUpdate} and \texttt{fallbackUpdate}	&			\\
								& 1: Add multiples of identity to first Hessian					&			\\
								& \phantom{1: } [not implemented yet]										&\\\hline
\end{longtable}

%---------------------------------------------------------------------72
\subsection{List of algorithmic parameters}
%---------------------------------------------------------------------72
\begin{longtable}[c]{lll}
Name 							& Symbol/Meaning												& Default			\\\hline\hline
%
\texttt{opttol}					& $\epsilon_{\textup{opt}}$ 									& 1.0e-5	\\\hline
%
\texttt{nlinfeastol}			& $\epsilon_{\textup{feas}}$									& 1.0e-5	\\\hline
%
\texttt{eps}					& machine precision												& 1.0e-16	\\\hline
%
\texttt{inf}					& $\infty$														& 1.0e20	\\\hline
%
\texttt{maxItQP}				& Maximum number of QP iterations per							& 5000		\\
								& SQP iteration (\texttt{int}>0)								&			\\\hline
%
\texttt{maxTimeQP}				& Maximum time in second for \qpOASES\ per						& 10000.0		\\
								& SQP iteration (\texttt{double}>0)								&			\\\hline
%
\texttt{maxConsecSkippedUpdates}& Maximum number of skipped updates 							& 100		\\
								& before Hessian is reset (\texttt{int}>0)						&			\\\hline
%
\texttt{maxLineSearch}			& Maximum number of line search iterations (\text{int}>0)		& 20		\\\hline
%
\texttt{maxConsecReducedSteps}	& Maximum number of reduced steps 								& 100		\\
								& before restoration phase is invoked (\text{int}>0)			&			\\\hline
%
\texttt{hessMemsize}			& Size of Hessian memory (\texttt{int}>0)						& 20		\\\hline
%
\texttt{maxSOCiter}				& Maximum number of second-order correction steps				& 3			\\\hline
%\texttt{colEps}				&
%%
%\texttt{colTau1}				&
%%
%\texttt{colTau2}				&
%%
%\texttt{iniHessDiag}			&
%%
%\texttt{hessDampFac}			&
%\texttt{maxSOCiter}
%\texttt{gammaTheta}
%\texttt{gammaF}
%\texttt{kappaSOC}
%\texttt{kappaF}
%\texttt{thetaMax}
%\texttt{thetaMin}
%\texttt{delta}
%\texttt{sTheta}
%\texttt{sF}
%\texttt{kappaMinus}
%\texttt{kappaPlus}
%\texttt{kappaPlusMax}
%\texttt{deltaH0}
%\texttt{eta}
\end{longtable}

%%%%%%%%%%%%%%%%%%%%%%%%%%%%%%%%%%%%%%%%%%%%%%%%%%%%%%%%%%%%%%%%%%%%%%72
\section{Output}
%%%%%%%%%%%%%%%%%%%%%%%%%%%%%%%%%%%%%%%%%%%%%%%%%%%%%%%%%%%%%%%%%%%%%%72

%%%%%%%%%%%%%%%%%%%%%%%%%%%%%%%%%%%%%%%%%%%%%%%%%%%%%%%%%%%%%%%%%%%%%%72
\section{Notes for developers}
%%%%%%%%%%%%%%%%%%%%%%%%%%%%%%%%%%%%%%%%%%%%%%%%%%%%%%%%%%%%%%%%%%%%%%72

\bibliographystyle{plain}
\bibliography{references.bib}
\end{document}


